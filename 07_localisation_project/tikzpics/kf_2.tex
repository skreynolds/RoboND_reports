\begin{tikzpicture}[auto, node distance=2cm]

    % Input
    \node [input, name=input] {};
    
    % b gain
    \node [block, right of=input] (b) {$b$};
    
    % sum element
    \node [sum, right of=b,
    pin={[pinstyle]above:$\omega_j$},
            node distance=2cm] (sum) {};
    
    % Two silent nodes in the middle
    \node [output, right of=sum] (x1) {};
    \node [output, right of=x1] (x2) {};
	 
	% Output node
    \node [output, right of=x2] (output) {};
    
    % Nodes in the feedback loop
    \node [block, below of=x2] (delay) {$z^{-1}$};
	\node [block, below of=x1] (a) {$a$};
	
	% Gain on the xj signal
	\node [block, right of=output, node distance=1cm] (h) {$h$};
	
	% sum node
	\node [sum, right of=h,
	    pin={[pinstyle]above:$\nu_j$},
	            node distance=2cm] (sum2) {};
	            
	% measurement output
	\node [output, right of=sum2] (output2) {};
	
    % Once the nodes are placed, connecting them is easy. 
    \draw [draw,->] (input) -- node {$u_j$} (b);
    \draw [->] (b) -- (sum);
    \draw [-] (sum) -- (x1);
    \draw [-] (x1) -- (x2);
    \draw [-] (x2) -- node [name=y] {$x_j$}(output);
    \draw [->] (y) |- (delay);
    \draw [->] (delay) -- (a);
    \draw [->] (a) -| node[pos=0.99] {$+$} 
        node [near end] {$a \cdot x_{j-1}$} (sum);
    
    % measurement section
    \draw [->] (output) -- (h);
    \draw [->] (h) -- (sum2);
    \draw [->] (sum2) -- node [name=y2] {$z_j$}(output2);
\end{tikzpicture}